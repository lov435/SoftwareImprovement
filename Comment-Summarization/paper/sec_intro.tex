\section{Introduction}

% talk about bug prioritization and talk about how interaction traces have
% rarely been considered for this purpose.

%
% stack-trace is important data debugging, in particular, for software
% in production.
%
% it is important to understand bugs, in particular, the context of
% the bugs, a characteristics of bugs

Continuous monitoring of deployed software usage is now a standard approach in
industry. Developers leverage usage data to discover and correct faults,
performance bottlenecks, or inefficient user interface design.  This practice
has led to a debugging methodology called ``debugging in the large'', a
postmortem analysis of large amount of usage data to recognize patterns of
bugs~\cite{Han:2012:PDL:2337223.2337241, glerum2009debugging}. For instance,
Arnold et al.\ use application stack traces to group processes exhibiting
similar behavior called ``process equivalence classes'', and identify what
differentiate these classes with the aim to discover the root cause of the bugs
associated with the stack traces~\cite{arnold2007stack}. Han et al.\ clusters
stack traces and recognize patterns of stack traces to discover impactful
performance bugs~\cite{Han:2012:PDL:2337223.2337241}.

%Following the above, we begin our work with the following observation.
Software-as-a-service applications often gather monitoring data at the service
host, while user-installed client software collects relevant traces (or logs)
periodically at the user's machines and transferred them from users'
machines to a server. The granularity and format of the collected data (e.g.,
whether the format of the data is a raw/log form or as a set of derivative
metrics) depend on the specific application and deployment. Two types of data
commonly collected via monitoring include {\em software exceptions}, containing
a stack traces from software faults that occur in production, and {\em
interaction traces}, containing details of user interactions with the
software's interface.


By utilizing datasets that contain both of these two types of data, we can provide 
a novel perspective on interpreting frequently occurring stack traces resulting from software exceptions by
modeling them in concert with the user interactions with which they co-occur.
Our approach probabilistically represents stack traces and their interaction context
for the purpose of increasing developer understanding of specific software
faults and the contexts in which they appear. Over time, this understanding can
help developers to reproduce exceptions, to prioritize software crash reports
based on their user impact, or to identify specific user behaviors that tend to
trigger failures.  Existing works attempt to empirically characterize software
crash reports in application domains like operating systems, networking
software, and open source software
applications~\cite{Yin:2010:TUB:1823844.1823849, Chou:2001:ESO:502059.502042,
Li:2006:TCE:1181309.1181314, Lu:2008:LMC:1353535.1346323}, but none have used
interaction traces containing stack traces for the purpose of fault characterization 
debugging.

%% why it is innovative to use interaction or behavior data to understand bugs
%Much software requires human interaction, such as, any software with user
%interface. During users' interaction with a piece of software, a software fault
%may happen, and it leads to an exception, followed by a software crash.  During
%the process, many messages are written to log traces. These messages often
%include stack traces and messages representing the interactions between the
%users and the software.  Capturing the interactions between the user and her
%software and the stack traces, we can analyze the interactions and establish in
%which context the exception may occur, useful information to the developers.
%Noting that developers' interaction traces with IDE have not been used for
%understanding software bugs, we posit that a hierarchical analysis can
%effectively lead to understanding the context of exceptions, and potentially
%software bugs.

% Why hierarchical models?
Interaction traces can be challenging to analyze. First, the logged
interactions are typically low-level, corresponding to most mouse clicks and key
presses available in the software application, and therefore the raw number of
interactions in these traces can be large --- containing millions of messages
from different users. Second, for complex software applications, there are
often multiple reasonable interaction paths to accomplish a specific high-level
task while interaction traces that lead to different tasks can share shorter
but common interaction paths.  To address these two challenges of scale and of
uncertainty in interpreting interaction traces, we posit that probabilistic
dimension reduction techniques that can extract frequent patterns from the
low-level interaction data are the right choice to analyze interaction traces.

Topic models are such a dimensionality reduction technique with the capacity to
discover complex latent thematic structures. Typically applied to large textual
document collections, such models can naturally capture the uncertainty in
software interaction data using probabilistic assumptions; however, in cases
where the interaction traces are particularly complex, e.g., in complex software
applications such as IDEs or CAD tools, applying typical topic models may still
result in a large topic space that is difficult to interpret. The special class
of hierarchical topic models encodes a tree of related topics, enabling further
reduction in complexity and dimensionality of the original interaction data and
improving the interpretability of the model. We apply a hierarchical topic
modeling technique, called the Nested Hierarchical Dirichlet Process
(NHDP)~\cite{6802355} to combine interaction traces and stack traces gathered
from a complex software application into a single, compact representation. The
NHDP discovers a hierarchical structure of usage events that has the following
characteristics:

\begin{itemize}
\item provides an interpretable summary of the user interactions that commonly co-occur with specific stack traces;
\item allows for differentiating the strength of the relationship between specific interaction trace messages and a stack trace; and
\item enables locating specific interactions that have co-occurred with numerous runtime errors.
\end{itemize}

\noindent In addition, as a Bayesian non-parametric modeling technique, NHDP
has an additional advantage. It allows the model to grow structurally as it
observes more data. Specifically, instead of imposing a fixed set of topics or
hypotheses about the relationship of the topics, the model grows its hierarchy
to fit the data, i.e., to ``let the data
speak''~\cite{Blei:2010:NCR:1667053.1667056}. This is beneficial in modeling the
datasets of interest since users' interaction with software
changes as the software does, e.g., by adding new features or
removing (or introducing) new bugs. 

%If an interaction (a command or an event)  is commonly associated with a stack
% trace, then the relationship can be further explored to diagnose the problem.

The main contributions of this paper are as follows:
\begin{itemize}
    \item We apply a hierarchical topic model to a large collection of
      interaction and stack trace data produced by ABB RobotStudio, a popular
      robot programming platform developed at ABB Inc, and examine how
      effective it extracts latent thematic structures of the dataset and how
      well the structure depicts a context for exceptions occurring
      during the production use of RobotStudio. 

    \item We are first to propose the idea of grouping users' IDE interaction traces with
        stack traces hierarchically and probabilistically into ``clusters''.
        These ``clusters'' provide user interaction contexts of stack traces.  Since a
        stack trace may be the result of multiple different interaction contexts, this
        approach associates a stack trace with its contexts probabilistically. 

\end{itemize}

We organize the remainder of this paper as follows.
Section~\ref{sec:background} introduces the types of interaction and stack
trace data we use and how we prepare these data sources for topic modeling. We
describe the hierarchical topic modeling technique and its application to
software interaction and crash data in Section~\ref{sec:nhdp}. We apply the
modeling technique to the large RobotStudio dataset and provide an evaluation
in Section~\ref{sec:eval}. Our work is not without threats to its validity, which we discuss in
Section~\ref{sec:threat}. In
Section~\ref{sec:related-work}, we describe relevant related research and
conclude this paper in Section~\ref{sec:end}.
