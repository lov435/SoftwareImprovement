\begin{abstract}

    Traces of user interactions with a software system, captured in production,
    are commonly used as an input source for user experience testing. In this
    paper, we present an alternative use, introducing a novel approach of
    modeling user interaction traces enriched with another type of data gathered
    in production - software fault reports consisting of software exceptions and
    stack traces. The model described in this paper aims to improve developers'
    comprehension of the circumstances surrounding a specific software exception
    and can highlight specific user behaviors that lead to a high frequency of
    software faults.


    Modeling the combination of interaction traces and software crash reports to
    form an interpretable and useful model is challenging due to the complexity
    and variance in the combined data source. Therefore, we propose a
    probabilistic unsupervised learning approach, adapting the Nested
    Hierarchical Dirichlet Process, which is a Bayesian non-parametric
    hierarchical topic model originally applied to natural language data. This
    model infers a tree of topics, each of whom describes a set of commonly
    co-occurring commands and exceptions. The topic tree can be interpreted
    hierarchically to aid in categorizing the numerous types of exceptions and
    interactions. We apply the proposed approach to large scale datasets
    collected from the ABB RobotStudio software application, and evaluate it
    both numerically and with a small survey of the RobotStudio developers.


    \keywords{
        Stack Trace, Crash Report, Software Exception, Software Interaction
        Trace, Hierarchical Topic Model
    }

\end{abstract}
